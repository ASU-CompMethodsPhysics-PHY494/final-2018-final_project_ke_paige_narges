\documentclass[12pt]{article}
\usepackage[margin=1in]{geometry} 
\usepackage{amsmath,amsthm,amssymb,amsfonts,bm}
\usepackage{graphicx}
\newcommand{\Rey}{\mathit{Re}}
\newcommand{\Ra}{\mathit{Ra}}
\newcommand{\Ri}{\mathit{Ri}}
\newcommand{\Gr}{\mathit{Gr}}
\newcommand{\Pe}{\mathit{Pe}}
\newcommand{\Nu}{\mathit{Nu}}
\newcommand{\Ek}{\mathit{E}}
\renewcommand{\Pr}{\mathit{Pr}}
 

\begin{document}
\title{Dynamics in a Stratified Lid-driven Cavity}
\author{Paige Weisman,Narges Masoumi, Ke Wu}
\maketitle
 
\section{Problem Description}
Consider the flow in a rectangular cavity of depth $d$ and width $l$,
filled with a fluid of kinematic viscosity $\nu$, thermal diffusivity
$\kappa$, and coefficient of volume expansion $\beta$. The top
boundary of the cavity is driven horizontally at a constant speed $U$
and is maintained at a fixed temperature $T_\text{top}$. The bottom
boundary is maintained at a cooler fixed temperature $T_\text{bot}$,
whilst the two sidewalls are insulated. Gravity $g$ acts in the
downward vertical direction. The temperature deviation from the mean temperature,
$(T_\text{top}+T_\text{bot})/2$, is non-dimensionalized with $\Delta
T= T_\text{top}-T_\text{bot}$. Figure~\ref{schematic} shows a
schematic of the system. Using $d$ as the length scale and the viscous
time $d^2/\nu$ as the time scale, the non-dimensional governing
equations, employing the Boussinesq approximation, are
\begin{equation}  \label{goveq}
  \begin{aligned}
    \bm{u}_t+\bm{u}\cdot\nabla \bm{u} &= -\nabla p + \nabla^2\bm{u}
    + \Gr\,T\bm{e}_y\,, \quad   \nabla\cdot\bm{u}=0\,,\\ 
    T_t+\bm{u}\cdot\nabla T &= \frac{1}{\Pr} \nabla^2T\,.
  \end{aligned}
\end{equation}
The boundary conditions are:
\begin{equation}
\begin{aligned}
 & \text{top,    }  (x,y)=(x,0.5):       &&  (u,v,T)=(\Rey,0,0.5), \\
 & \text{bottom, }  (x,y)=(x,-0.5):      &&  (u,v,T)=(0,0,-0.5),\\
 & \text{sides,  }  (x,y)=(\pm0.5\gamma,y): &&  (u,v,T_x)=(0,0,0),
\end{aligned}
\end{equation}
where $\gamma=l/d$ is the aspect ratio, $\Gr=\Ri \Rey^{2}$, $\Pr$ is the Prandtl number,
$\Gr$ is the Grashof number and $\Ri$ is the Richardson number, which controls
the intensity of the buoyancy force. The global energy is used to
quantify the flow dynamics inside the cavity, which is defined as 
\begin{equation} \label{Ek}
\Ek = \frac{1}{\Rey^{2}}\int_{y=-0.5}^{y=0.5}\int_{x=-0.5}^{x=0.5}
(u^{2}+v^{2})dxdy\,.
\end{equation}


\begin{figure}[t]
\centering{\includegraphics[width=0.85\linewidth]{D2cavity_Schematics}}
\caption{Schematic of the system, indicating the non-dimensional
         coordinate system and boundary conditions. The top boundary moves to
         the right (positive $x$ direction) at constant speed $\Rey$. The
         colour map shows the linearly stratified temperature distribution
         (red is $T=0.5$, blue is $T=-0.5$, and white is $T=0$) when
         $\Rey=0$.}
\label{schematic}
\end{figure}

\section{Approach}
The governing equations \eqref{goveq} are discretized using a
spectral-collocation method in both spatial directions, which is
proposed by \cite{HuRa98}. Both velocity
components and pressure are approximated by Chebyshev polynomials of
the first kind with degree less than $M$, associated with the
Chebyshev--Gauss--Lobatto grid. Spatial differentiation is performed
via direct matrix-vector multiplication by the pseudo-spectral
differentiation matrix.
\noindent
The boundary condition for the $x$-velocity at the moving top wall,
$u(x,0.5)=\Rey$, is discontinuous where it meets the two vertical
sidewalls, which can lead to Gibbs phenomena when using a global
spectral method. In order to resolve this numerical issue, we
regularise the top wall boundary condition by using:
\begin{equation}
u(x,0.5) = \Rey\left[1-\exp{(-(1-4x^{2})/\delta)}\right].
\end{equation} 

\section{Objective}
In this project, we are going to fix the $\Rey = 2750$ and vary $Ri$
numbers.For small $Ri$ values, the flow will be steady state with
respect to time, but as we increase $Ri$ number, the flow inside the square cavity
is expected to lose its stability and become periodic or even
quasi-periodic. If we further increase $Ri$ number, the flow is expected
to become steady flow again this is due to the buoyancy force is the
dominant force which controls the flow dynamics. As we vary the
parameter $Ri$, the flow dynamics will change dramatically, there are
different types of bifurcations might occur, therefore, the numerical 
simulations not only can help us identify the different states of flow, 
but also can give us enough information that we can classify the bifurcations of the
flow as we increase the parameter $Ri$ number. The specific tasks are
following:
\begin{itemize}
\item{The code will be tested without tempture stratification, there are 
  many known solution for $2D$ lid driven cavity problem.}
\item{with $\bm{u}=0$ initial condition, simulate the flow for $Ri\in
  [0.1,2]$ with increment of $\Delta Ri=0.1$. For single case, record
  the energy time series for each time step.} 
\item{Use the solutions obtained in the previous step as initial
  conditions, do a refinement. Use $\Delta Ri=0.01$, in total there will
  be $190$ cases.}
\item{Compute the FFT for each time series, and plot the power spectal
  density diagram.}
\item{Use the power spectal density plot to find the critical $Ri$ where 
  bifurcations occur, and make the bifurcation diagram, which will be
  quantified by global energy.}
\item{For different flow states, make a $2D$ plot for a steady state
  solution, a movie for a limit cycle solution.}
\end{itemize}

\section{Appendix}
A pseudospectal collocation-Chebyshev method is implemented, each variable is expanded 
in the approximation space $P_{MN}$, composed of Chebyshev polynomials, 
$T_{m}$ and $T_{n}$ of degrees less or equal than $M$ and $N$ in $x, y$ directions.

\begin{equation}
T(x,y) =
  \sum_{m=0}^{M}\sum_{n=0}^{N}\hat{T}_{mn}T_{m}(2x)T_{n}(2y)
\end{equation}

\begin{equation}
u(x,y) = 
  \sum_{m=0}^{M}\sum_{n=0}^{M}\hat{u}_{mn}T_{m}(2x)T_{n}(2y)
\end{equation}

\begin{equation}
v(x,y) = 
  \sum_{m=0}^{M}\sum_{n=0}^{N}\hat{v}_{mn}T_{m}(2x)T_{n}(2y)
\end{equation}

\begin{equation}
p(x,y) = 
  \sum_{m=0}^{M}\sum_{n=0}^{M}\hat{p}_{mn}T_{m}(2x)T_{n}(2y)
\end{equation}

All the partial derivatives are evaluated directly by the Chebyshev differentiation matrix. 
Since the computational domain is $[-1/2,1/2]^{2}$, when we take partial derivatives 
by use Chebyshev differentiation matrix $D$, we need to use chain rule, for instance
\begin{equation}
\frac{\partial u}{\partial x} = 2Du
\end{equation}

\begin{equation}
\frac{\partial^{2} u}{\partial x^{2}} = 4D^{2}u
\end{equation}

Since we use the Chebyshev collocation method, so all the values are directly 
evaluated at the Chebyshev-Gauss-Lobatto collocation points. Also note that since 
the aspect ratio we choose is 1, so there is a factor of 2 when we take the first 
partial derivative, and there is a factor of 4 when we take the second partial derivative.\\

Temporal scheme:\\
\begin{equation}
\frac{3\vec{u}^{n+1} - 4\vec{u}^{n} + \vec{u}^{n-1}}{2\delta t} =
- \nabla p^{n+1} - 2NL(\vec{u}^{n})
+ NL(\vec{u}^{n-1}) - \triangle \vec{u}^{n+1} + GrT^{n+1}\vec{e}_{y} 
\end{equation}

\begin{equation}
\frac{3T^{n+1} - 4T^{n}+T^{n-1}}{2\delta t} =
- 2NL(\vec{u}^{n},T^{n})
+  NL(\vec{u}^{n-1},T^{n-1}) + \frac{1}{Pr} \triangle T^{n+1}
\end{equation}

The time integration used is second order accurate and is based on a combination 
of Adams-Bashforth and backward differentiation formula (AB2/BDF) schemes.\\
In order to improve the stability, the viscous term need to be written as

\begin{equation}
L(\vec{u})=\Delta \vec{u} = 
 \nabla(\nabla \cdot \vec{u})-\nabla \times (\nabla \times \vec{u})=
-\nabla \times (\nabla \times \vec{u})
\end{equation}

which is due to divergence free.\\
\\
\centerline{Code Implementation}
Every quantity is evaluated directly on the Chebyshev-Gauss-Lobatto grid points, 
Let DX,DY denote the Chebyshev differentiation matrix with dimension $(M+1) \times (N+1)$, 
and use $i,j$ to denote the indices start from 0 to $M$ and $N$ respectively. 
In our code, define $U(0:M,0:N)$, $V(0:M,0:N)$,$T(0:M,0:N)$, $P(0:M,0:N)$ 
as the velocity, 
temperature and pressure variables, $NLu(0:M,0:N)$, $NLv(0:M,0:N)$ as the nonlinear
terms in momentum equation,
and Define $NLT(0:M,0:N)$ as the nonlinear terms in temperature equation, 
$Lu(0:M,0:N)$, $Lv(0:M,0:N)$ as the viscous terms in the momentum equation. 
So these nonlinear terms can be computed as following:
let $asp = 2$
\begin{align}
NLu = u\frac{\partial u}{\partial x} + v\frac{\partial u}{\partial y}
=asp*U * DXU + 2V * UDY^{T}
\\
NLv = u\frac{\partial v}{\partial x} + v\frac{\partial v}{\partial y}
=asp*U*DXV + 2V*VDY^{T}
\\
NLT = u\frac{\partial T}{\partial x} + v\frac{\partial T}{\partial y}
=asp*U*DXT + 2V*TDY^{T}
\\
Lu  = \frac{\partial ^2u}{\partial y^{2}} - \frac{\partial ^2v}{\partial y\partial x}
=4\{U(DY^{2})^{T}\} - 2*asp*\{(DXV)DY^{T}\}
\\
Lv  = \frac{\partial ^2v}{\partial x^{2}} - \frac{\partial ^2u}{\partial x\partial y}
=asp^{2}*\{(DX^{2}V)\} - asp*2*\{DX(UDY^{T})\}
\end{align}
\textbf{The Improved Projection Scheme:}

\begin{description}
\item[step 1:] compute the temperature $T^{n+1}$ from temperature equation\\
\begin{equation}
(\Delta -\frac{3Pr}{2\delta t}) T^{n+1} = Pr \{ 2NL(\vec{u}^{n},T^{n}) 
- NL(\vec{u}^{n-1},T^{n-1}) - \frac{4T^{n}-T^{n-1}}{2\delta t} \}
\end{equation}

with boundary conditions: \\
  $ T(x, 1/2) = 0.5$,
  $ T(x,-1/2) =-0.5$,
  $ \frac{\partial T}{\partial n}(-1/2,y) = 0 $,
  $ \frac{\partial T}{\partial n}( 1/2,y) = 0 $.

Formulation of the linear systems:\\
The indices $i,j$ are in the sets $\{0,1,2,...M\}$ and $\{ 0,1,2,...,N\}$ respectively.
\begin{align}
  \frac{\partial^{2} T^{n+1}}{\partial x^{2}}
 +\frac{\partial^{2} T^{n+1}}{\partial y^{2}}
 -\frac{3Pr}{2\delta t}T^{n+1}  = RHS \\
RHS = Pr \{ 2NL(\vec{u}^{n},T^{n}) - NL(\vec{u}^{n-1},T^{n-1}) 
  - \frac{4T^{n}-T^{n-1}}{2\delta t} \}
\end{align}

\begin{multline}
  asp^{2}\sum_{k=0}^{M}DX^{2}_{i,k}T_{k,j}^{n+1} +
  4\sum_{k=0}^{N}T_{i,k}^{n+1}(DY^{2})^{T}
  - \frac{3Pr}{2\delta t}T_{i,j}^{n+1} = RHS_{i,j}\\
  asp^{2}\sum_{k=1}^{M-1}DX^{2}_{i,k}T_{k,j}^{n+1}  
 +asp^{2}*DX^{2}_{i,0}T_{0,j}^{n+1} 
 +asp^{2}*DX^{2}_{i,M}T_{M,j}^{n+1}+\\
  4\sum_{k=1}^{N+1}T_{i,k}^{n+1}(DY^{2})^{T}_{k,j}
  +4T_{i,0}^{n+1}(DY^{2})^{T}_{0,j}
  +4T_{i,N}^{n+1}(DY^{2})^{T}_{N,j}
  -\frac{3Pr}{2\delta t}T_{i,j}^{n+1} = RHS_{i,j}
\end{multline}

\begin{equation}
  T_{0,j}^{n+1} =
-\sum_{k=1}^{M-1}\frac{DX_{MM}DX_{0k}-DX_{0M}DX_{Mk}}
  {DX_{00}DX_{MM}-DX_{M0}DX_{0M}} T_{k,j}^{n+1}
\end{equation}

\begin{equation}
  T_{M,j}^{n+1} =
-\sum_{k=1}^{M-1}\frac{DX_{M0}DX_{0k}-DX_{00}DX_{Mk}}
  {DX_{M0}DX_{0M}-DX_{00}DX_{MM}} T_{k,j}^{n+1}
\end{equation}

Note that the left and right temperature values at boundary points are expressed as 
a linear combination of interior points. Plug $T_{0,j},T_{M,j}$ into the linear equation, 
we obtain the following linear system:\\
\begin{multline}
  asp^{2}\sum_{k=1}^{M-1}DX^{2}_{i,k}T_{k,j}^{n+1}
-asp^{2}*DX^{2}_{i,0}\sum_{k=1}^{M-1}\frac{DX_{MM}DX_{0k}-DX_{0M}DX_{Mk}}
  {DX_{00}DX_{MM}-DX_{M0}DX_{0M}} T_{k,j}^{n+1}\\
-asp^{2}*DX^{2}_{i,M}\sum_{k=1}^{M-1}\frac{DX_{M0}DX_{0k}-DX_{00}DX_{Mk}}
  {DX_{M0}DX_{0M}-DX_{00}DX_{MM}} T_{k,j}^{n+1}
  +4\sum_{k=1}^{N-1}T_{i,k}^{n+1}(DY^{2})^{T}_{k,j}
  -\frac{3Pr}{2\delta t}T_{i,j}^{n+1} \\
  =RHS_{i,j} -4T_{i,0}^{n+1}(DY^{2})^{T}_{0,j} -
  4T_{i,N}^{n+1}(DY^{2})^{T}_{N,j}
\end{multline}
where the indices $i,j$ start from $\{1,2,3,...M-1\}$ and
$\{1,2,3,...,N-1\}$.

\item[step 2:] compute the preliminary presssure $P^{*}$\\
\begin{equation}
\triangle {p}^{*} = \nabla \cdot  \{-2NL(\vec{u}^{n}) + NL(\vec{u}^{n-1}) + GrT^{n+1}\vec{e}_{y} \}
\end{equation}
 with boundary conditions:\\
\begin{equation}
\frac{\partial {p}^{*}}{\partial{n}}= \vec{n} \cdot \{\frac{-3\vec{w}+4\vec{u}^{n}-\vec{u}^{n-1}}{2\delta t}
      -2NL(\vec{u}^{n}) + NL(\vec{u}^{n-1}) + GrT^{n+1}\vec{e}_{y} + 2L(\vec{u}^{n})-L(\vec{u}^{n-1}) \}
\end{equation}
where $\vec{w}$ is the boundary conditions of the velocity field.\\
Formation of the linear system:
\begin{equation}
asp^{2}\sum_{k=0}^{M}DX^{2}_{i,k}P^{*}_{k,j} + 4\sum_{k=0}^{N}P^{*}_{i,k}(DY^{2})^{T}_{k,j}
= RHS_{i,j}\\
\end{equation}

\begin{equation}
RHS = asp*DX\{-2NL(u^{n}+NL(u^{n-1}) \}
+2\{-2NL(v^{n})+NL(v^{n-1})+GrT^{n+1} \} DY^{T}
\end{equation}

\begin{align}
  asp^{2}\sum_{k=1}^{M-1}DX^{2}_{i,k}P^{*}_{k,j}
+ asp^{2}*DX^{2}_{i,0}P^{*}_{0,j}+asp^{2}*DX^{2}_{i,M}P^{*}_{M,j}\\
+ 4\sum_{k=1}^{N-1}P^{*}_{i,k}(DY^{2})^{T}_{k,j}
+ 4P^{*}_{i,0}(DY^{2})^{T}_{0,j}
+ 4P^{*}_{i,N}(DY^{2})^{T}_{N,j}
= RHS_{i,j}\\
\end{align}
\begin{align}
P^{*}_{0,j} =
\frac{\frac{1}{asp}rhs1_{0,j}DX_{MM}-\frac{1}{asp} rhs1_{M,j}DX_{0M}}{DX_{00}DX_{MM}-DX_{0M}DX_{M0}}
-\sum_{k=1}^{M-1}\frac{DX_{MM}DX_{0k}-DX_{0M}DX_{Mk}}
{DX_{00}DX_{MM}-DX_{M0}DX_{0M}} P^{*}_{k,j}\\
P^{*}_{M,j} =
\frac{\frac{1}{asp} rhs1_{0,j}DX_{M0}-\frac{1}{asp}rhs1_{M,j}DX_{00}}{DX_{M0}DX_{0M}-DX_{00}DX_{MM}}
-\sum_{k=1}^{M-1}\frac{DX_{M0}DX_{0k}-DX_{00}DX_{Mk}}
{DX_{M0}DX_{0M}-DX_{00}DX_{MM}} P^{*}_{k,j}\\
P^{*}_{i,0} =
\frac{\frac{1}{2}rhs2_{i,0}DY^{T}_{NN}-\frac{1}{2} rhs2_{i,N}DY^{T}_{N0}}{DY^{T}_{00}DY^{T}_{NN}-DY^{T}_{0N}DY^{T}_{N0}}
-\sum_{k=1}^{N-1}P^{*}_{i,k}\frac{DY^{T}_{NN}DY^{T}_{k0}-DY^{T}_{N0}DY^{T}_{kN}}
{DY^{T}_{00}DY^{T}_{NN}-DY^{T}_{0N}DY^{T}_{N0}} \\
P^{*}_{i,N} =
\frac{\frac{1}{2}rhs2_{i,0}DY^{T}_{0N}-\frac{1}{2} rhs2_{i,N}DY^{T}_{00}}{DY^{T}_{0N}DY^{T}_{N0}-DY^{T}_{00}DY^{T}_{NN}}
-\sum_{k=1}^{N-1}P^{*}_{i,k}\frac{DY^{T}_{0N}DY^{T}_{k0}-DY^{T}_{00}DY^{T}_{kN}}
{DY^{T}_{0N}DY^{T}_{N0}-DY^{T}_{00}DY^{T}_{NN}}
\end{align}

Plug the boundary $P^{*}$  values into the poisson equation, the linear system can be written as following:\\
\begin{align}
asp^{2}\sum_{k=1}^{M-1}DX^{2}_{i,k}P^{*}_{k,j}
-asp^{2}DX^{2}_{i,0} \sum_{k=1}^{M-1}\frac{DX_{MM}DX_{0k}-DX_{0M}DX_{Mk}}
{DX_{00}DX_{MM}-DX_{M0}DX_{0M}} P^{*}_{k,j}\\
-asp^{2}DX^{2}_{i,M}\sum_{k=1}^{M-1}\frac{DX_{M0}DX_{0k}-DX_{00}DX_{Mk}}
{DX_{M0}DX_{0M}-DX_{00}DX_{MM}} P^{*}_{k,j}\\
+4\sum_{k=1}^{N-1}P^{*}_{i,k}(DY^{2})^{T}_{k,j}
-4\sum_{k=1}^{N-1}P^{*}_{i,k}\frac{DY^{T}_{NN}DY^{T}_{k0}-DY^{T}_{N0}DY^{T}_{kN}}
{DY^{T}_{00}DY^{T}_{NN}-DY^{T}_{0N}DY^{T}_{N0}}(DY^{2})^{T}_{0,j}\\
-4\sum_{k=1}^{N-1}P^{*}_{i,k}\frac{DY^{T}_{0N}DY^{T}_{k0}-DY^{T}_{00}DY^{T}_{kN}}
{DY^{T}_{0N}DY^{T}_{N0}-DY^{T}_{00}DY^{T}_{NN}}(DY^{2})^{T}_{N,j}\\
= RHS_{i,j}-asp*DX^{2}_{i,0}\frac{rhs1_{0,j}DX_{MM} - rhs1_{M,j}DX_{0M}}{DX_{00}DX_{MM}-DX_{0M}DX_{M0}}\\
-asp*DX^{2}_{i,M}\frac{rhs1_{0,j}DX_{M0}-rhs1_{M,j}DX_{00}}{DX_{M0}DX_{0M}-DX_{00}DX_{MM}}\\
-2*(DY^{2})^{T}_{0,j}\frac{rhs2_{i,0}DY^{T}_{NN}- rhs2_{i,N}DY^{T}_{N0}}{DY^{T}_{00}DY^{T}_{NN}-DY^{T}_{0N}DY^{T}_{N0}}\\
-2*(DY^{2})^{T}_{N,j}\frac{rhs2_{i,0}DY^{T}_{0N}- rhs2_{i,N}DY^{T}_{00}}{DY^{T}_{0N}DY^{T}_{N0}-DY^{T}_{00}DY^{T}_{NN}}
\end{align}


\item[step 3:] compute the predictor velocity field $\vec{u}^{*}$ from the momentum equation
\begin{equation}
\frac{3\vec{u}^{*}-4\vec{u}^{n}+\vec{u}^{n-1}}{2\delta t} +2NL(\vec{u}^{n})-NL(\vec{u}^{n-1})
  =-\nabla p^{*}+\Delta \vec{u}^{*}+GrT^{n+1}\hat{e}_{y}
\end{equation}
Re-write the above equation and group all the $\vec{u}^{*}$ terms.
\begin{equation}
(\Delta -\frac{3}{2\delta t})\vec{u}^{*}= \nabla p^{*} + 2NL(\vec{u}^{n}) - NL(\vec{u}^{n-1}) 
  -GrT^{n+1}\vec{e}_{y} - \frac{4\vec{u}^{n}-\vec{u}^{n-1}}{2\delta t}
\end{equation}
with prescribed Dirichlet boundary conditions. Write the above equation in component form $\vec{u}=(u,v)$.\\
\begin{equation}
asp^{2}\sum_{k=0}^{M}DX^{2}_{i,k}u^{*}_{k,j} + 4\sum_{k=0}^{N}u^{*}_{i,k}(DY^{2})^{T}_{k,j}
-\frac{3}{2\delta t}u^{*}_{i,j}= RHS1_{i,j}\\
\end{equation}
\begin{equation}
asp^{2}\sum_{k=0}^{M}DX^{2}_{i,k}v^{*}_{k,j} + 4\sum_{k=0}^{N}v^{*}_{i,k}(DY^{2})^{T}_{k,j}
-\frac{3}{2\delta t}v^{*}_{i,j}= RHS2_{i,j}\\
\end{equation}
\begin{align}
RHS1 = \{ \frac{\partial p^{*}}{\partial x} +2NL(u^{n})-NL(u^{n-1})-
  \frac{4u^{n}-u^{n-1}}{2\delta t} -F^{n+1}_{x}  \}\\
= \{ asp*DXP^{*} +2NLU-NLUold-\frac{4U-Uold}{2\delta t} \}_{i,j}\\
RHS2 = \{ \frac{\partial p^{*}}{\partial y} +2NL(v^{n})-NL(v^{n-1})-
  \frac{4v^{n}-v^{n-1}}{2\delta t} -F^{n+1}_{y} -GrT^{n+1} \}\\
= \{ 2P^{*}(DY)^{T} +2NLV-NLVold-\frac{4V-Vold}{2\delta t} -GrT^{n+1} \}
\end{align}
In this step we impose the exact boundary values for the velocity components, 
and then reduce the system as $M-1$ by $N-1$ linear system.
\begin{multline}
\sum_{k=1}^{M-1}DX^{2}_{i,k}u^{*}_{k,j} + \sum_{k=1}^{N-1}u^{*}_{i,k}(DY^{2})^{T}_{k,j}
-\frac{3}{2\delta t}u^{*}_{i,j}\\
= RHS1_{i,j}-asp^{2}DX^{2}_{i,0}W_{0,j}-asp^{2}DX^{2}_{i,M}W_{M,j}
-4W_{i,0}(DY^{2})^{T}_{0,j}-4W_{i,N}(DY^{2})^{T}_{N,j}
\end{multline}
\begin{multline}
\sum_{k=1}^{M-1}DX^{2}_{i,k}v^{*}_{k,j} + \sum_{k=1}^{N-1}v^{*}_{i,k}(DY^{2})^{T}_{k,j}
-\frac{3}{2\delta t}v^{*}_{i,j}\\
= RHS2_{i,j}-asp^{2}DX^{2}_{i,0}W_{0,j}-asp^{2}DX^{2}_{i,M}W_{M,j}
-4W_{i,0}(DY^{2})^{T}_{0,j}-4W_{i,N}(DY^{2})^{T}_{N,j}\\
\end{multline}

\item[step 4:] correct the preliminary pressure and preliminary velocity through 
evaluating an intermediate variable $\phi$:\\

\begin{equation}
\frac{3\vec{u}^{n+1}-3\vec{u}^{*}}{2\delta t} = -\nabla (p^{n+1}-{p}^{*})
\end{equation}
Apply the continuity equation and define the intermediate variable $\phi$ as $\phi 
= \frac{2\delta t}{3}(p^{n+1}-{p}^{*})$, we obtain following equation:\\
\begin{equation}
\Delta \phi = \nabla \cdot \vec{u}^{*}
\end{equation}
with pure Neumann boundary condition:\\

\begin{equation}
\frac{\partial \phi}{\partial n} =0
\end{equation}

\begin{align}
RHS = \frac{\partial u^{*}}{\partial x} + \frac{\partial v^{*}}{\partial y}\\
RHS = (2DXU^{*} + 2V^{*}(DY)^{T})
\end{align}

\begin{equation}
4\sum_{k=0}^{M}DX^{2}_{i,k}\phi_{k,j} + 4\sum_{k=0}^{N}\phi_{i,k}(DY^{2})^{T}_{k,j}= RHS_{i,j}
\end{equation}
Express the boundary values of $\phi$ in terms interior points values by the Neumann boundary condition.\\
\begin{align}
\phi_{0,j} = 
-\sum_{k=1}^{M-1}\frac{DX_{MM}DX_{0k}-DX_{0M}DX_{Mk}}
{DX_{00}DX_{MM}-DX_{M0}DX_{0M}} \phi_{k,j}\\
\phi_{M,j} = 
-\sum_{k=1}^{M-1}\frac{DX_{M0}DX_{0k}-DX_{00}DX_{Mk}}
{DX_{M0}DX_{0M}-DX_{00}DX_{MM}} \phi_{k,j}\\
\phi_{i,0} = 
-\sum_{k=1}^{N-1}\phi_{i,k}\frac{DY^{T}_{NN}DY^{T}_{k0}-DY^{T}_{N0}DY^{T}_{kN}}
{DY^{T}_{00}DY^{T}_{NN}-DY^{T}_{0N}DY^{T}_{N0}} \\
\phi_{i,N} = 
-\sum_{k=1}^{N-1}\phi_{i,k}\frac{DY^{T}_{0N}DY^{T}_{k0}-DY^{T}_{00}DY^{T}_{kN}}
{DY^{T}_{0N}DY^{T}_{N0}-DY^{T}_{00}DY^{T}_{NN}}
\end{align}
Plug in the boundary condition into the above equation, we can form the linear system for the $\phi$ equation.
\begin{align}
  asp^{2}\sum_{k=1}^{M-1}DX^{2}_{i,k}\phi_{k,j}
  -asp^{2}DX^{2}_{i,0}\sum_{k=1}^{M-1}\frac{DX_{MM}DX_{0k}-DX_{0M}DX_{Mk}}
{DX_{00}DX_{MM}-DX_{M0}DX_{0M}} \phi_{k,j} \\
  -asp^{2}DX^{2}_{i,M}\sum_{k=1}^{M-1}\frac{DX_{M0}DX_{0k}-DX_{00}DX_{Mk}}
{DX_{M0}DX_{0M}-DX_{00}DX_{MM}} \phi_{k,j}\\
  + asp^{2}\sum_{k=1}^{N-1}\phi_{i,k}(DY^{2})^{T}_{k,j}
  -asp^{2}\sum_{k=1}^{N-1}\frac{DY^{T}_{NN}DY^{T}_{k0}-DY^{T}_{N0}DY^{T}_{kN}}
{DY^{T}_{00}DY^{T}_{NN}-DY^{T}_{0N}DY^{T}_{N0}}(DY^{2})^{T}_{0,j}\\
  -asp^{2}\sum_{k=1}^{N-1}\phi_{i,k}\frac{DY^{T}_{0N}DY^{T}_{k0}-DY^{T}_{00}DY^{T}_{kN}}
{DY^{T}_{0N}DY^{T}_{N0}-DY^{T}_{00}DY^{T}_{NN}}(DY^{2}_{N,j})\\
= RHS_{i,j}
\end{align}

\item[step 5:] finally correct the pressure field on the entire domain
and update velocity on the entire  domain.\\
\begin{align}
p^{n+1}={p}^{*} + \frac{3}{2\delta t}\phi  \\
\vec{u}^{n+1} = \vec{u}^{*} - \nabla \phi
\end{align}

\end{description}
\bibliography{final_project.bib}
\bibliographystyle{plain}
\end{document}
